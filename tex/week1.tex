\cardfrontfoot{Week 1: Intro}

\begin{flashcard}[Experiment]{Wilhem Roux' embryos}
	Wilhem Roux showed that if one of two embryonic frog cells were killed, the result would be a half-complete frog. However, he did not detach the dead cell, which would have resulted in a normal frog. 
\end{flashcard}

\begin{flashcard}[Experiment]{Hans Drierich Sea Urchins}
	Hans Drierich showed that 4 Sea Urchin embryo cells could grow into separate organisms \textbf{IF} they were separated, contradicting Roux' experiment where he punctured one of two cells.
\end{flashcard}

\begin{flashcard}[Definition]{Pattern formation}
	One or multi-dimensional gradients of extracellular ligands, that cause cell differentiation depending on concentration. 
\end{flashcard}

\begin{flashcard}[Definition]{Types of regulatory signalling and their range}
    Gap junctions, only between adjacent cells \\
    Surface protein interactions, adjacent cells \\
    Diffusion of ligands, any two cells \\
\end{flashcard}

\begin{flashcard}[Definition]{Maternal factors}
	Proteins and mRNA supplied by the mother to the zygote. 
\end{flashcard}

\begin{flashcard}[Definition]{Mosaic vs regulatory development}
	\textbf{Mosaic}: two cells contain different growth factors (proteins, RNA) and thus take different paths \\
	\textbf{Regulatory}: two cells are identical, but receive different extracellular signalling and thus take different paths
\end{flashcard}

\begin{flashcard}[Definition]{Cell fate, specification and determination}
	\textbf{Fate}: normal developmental path for cell, either differentiation or apoptosis \\
	\textbf{Specification}: a specified cell has been given instructions, but can change path unless determined \\
	\textbf{Determination}: a determined cell cannot change its fate
\end{flashcard}

\begin{flashcard}[Definition]{Gastrulation}
	The process in which germ layers are formed. Organism starts to resemble itself. 
\end{flashcard}

\begin{flashcard}[Definition]{Cleavage}
	When the zygote divides without increasing the size of cells, resulting in a blastula. 
\end{flashcard}

\begin{flashcard}[Definition]{Blastula and blastocyst}
	Occurs after cleavage, and consists of a single, hollowed layer of cells. Called blastocysts in vertebrates. 
\end{flashcard}

\begin{flashcard}[Germ layers]{Ectoderm}
	Outer layer - skin (or cuticle in insects) and nervous system
\end{flashcard}

\begin{flashcard}[Germ layers]{Mesoderm}
	Middle layer - muscle, heart, blood. In vertebrates also skeleton and kidney
\end{flashcard}

\begin{flashcard}[Germ layers]{Endoderm}
	Inner layer - epithelial layer of gut. In vertebrates also liver and lungs
\end{flashcard}
