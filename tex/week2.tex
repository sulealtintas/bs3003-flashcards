\cardfrontfoot{Week 2: \textit{C. elegans}}

\begin{flashcard}[Cell lineages]{Number of somatic and germ cells}
    959 somatic cells, around 2000 germ cells; invariant
\end{flashcard}

\begin{flashcard}[Cell lineages]{Number of C, D and E cells, PCD count}
    47 C-cells, 20 D-cells, 34 E-cells. 1 C-cell undergoes PCD.
\end{flashcard}

\begin{flashcard}[Cell lineages]{Number of AB and MS cells, PCD count}
    606 AB cells, an additional 116 undergo PCD. 252 MS cells, additional 14 undergo PCD. 
\end{flashcard}

\begin{flashcard}[Cell lineages]{$AB_a$ cells form..}
    Neurons, skin, anterior mesodermal pharynx. 
\end{flashcard}

\begin{flashcard}[Cell lineages]{$AB_p$ cells form..}
    Neurons, skin, specialised cells. 
\end{flashcard}

\begin{flashcard}[Cell lineages]{\textbf{MS} cells form..}
    Muscle cells, nerve cells, posterior mesodermal pharynx
\end{flashcard}

\begin{flashcard}[Cell lineages]{\textbf{E} cells form..}
    Gut cells (only one cell type)
\end{flashcard}

\begin{flashcard}[Cell lineages]{\textbf{C} cells form..}
    Skin cells, nerve cells and muscle cells
\end{flashcard}

\begin{flashcard}[Cell lineages]{\textbf{D} cells form..}
    Muscle cells (only one cell type)
\end{flashcard}

\begin{flashcard}[Cell lineages]{\textbf{P} cells form.. due to the presence of .. }
    Germ line cells, presence of P-granules
\end{flashcard}

\begin{flashcard}[Definition]{Use $P_1$ and AB as examples of autonomous and conditional modes of specification}
    The AB cell requires the $P_1$ to develop, and is therefore undergoes conditional specification. $P_1$ cell can develop on its own, autonomous. 
\end{flashcard}

\begin{flashcard}[Early cell fates]{SKN-1 does .. is inhibited by..}
    SKN-1 specifies the EMS cell. If not present, the EMS cell becomes a P2-like cell. Inhibited by PIE-1. 
\end{flashcard}
    
\begin{flashcard}[Maternal genes]{Role of PIE-1}
    \textbf{P}harynx and \textbf{I}ntestinal \textbf{E}xcess. Inhibits the activity of SKN-1 in the P2-cell, preventing it from adopting EMS fate. 
\end{flashcard}

\begin{flashcard}[Maternal genes]{Role of MEX-1}
    \textbf{M}uscle \textbf{Ex}cess. Prevents SKN-1 from entering ABa and ABp, which prevents them adopting the EMS fate. If mutated, they develop into muscle cells.
\end{flashcard}



% should be a few more, like the par genes...



\begin{flashcard}[Induction events]{The role of APX-1 and GLP-1}
    Required to give ABp its cell identity. GLP-1 is activated in ABp because the cell is in contact with P2; if ABp and ABa are swapped, ABa will be induced instead.
\end{flashcard}

\begin{flashcard}[Induction events]{Equivalents of APX-1 and GLP-1 in Drosophila}
    Delta (ligand) and Notch (receptor)
\end{flashcard}

\begin{flashcard}[Induction events]{Descendants of ABa are specified as pharyngeal precursor cells through...}
     Activation of the transcription factor PHA-4. The receptor is GLP-1, but the ligand is \textit{not} APX-1
\end{flashcard}

\begin{flashcard}[Induction events]{Activation of PHA-4 is caused by...}
    GLP-1 is the receptor, but the ligand in this case is \textbf{not} APX-1
\end{flashcard}

\begin{flashcard}[Induction events]{The specification of gut (E) cells is induced through...}
    Cell-cell interactions between EMS and P2.
    The molecule mom-2 is released from P2 and activates the mom-5 receptor in EMS, which downregulates pop-1 so that the E cell is formed.
\end{flashcard}

\begin{flashcard}[Induction events]{Consequences of high vs low amounts of pop-1}
    Active: MS cells, as pop-1 specifies MS fate \\
    Inactive: E cells
\end{flashcard}

\begin{flashcard}[Genes]{The molecules involved in inhibiting POP-1 are...}
    mom-2, mom-5, mom-4, wrm-1, lit-1
\end{flashcard}

\begin{flashcard}[Vulva formation]{Vulva cell names,, ancestor cell, total cells in developed vulva}
    Anchor cell, P3-P8 (not the same as germ line), descends from the ABp cell. Mature vulva contains 22 cells. 
\end{flashcard}

\begin{flashcard}[Vulva formation]{Determinant for primary cell fate, fate if mutated}
    LIN-3 from the anchor cell, which binds to LET-23 receptor. Results in repressing the LIN-12 receptor, which determines secondary cell fate. If mutated, all cells become tertiary, vulvaless.
\end{flashcard}

\begin{flashcard}[Vulva formation]{Determinant for secondary cell fate, fate if mutated}
    LIN-12 activated by ligands from the primary cell. Signal represses primary fate. If mutated, the two secondary cells become primary, multivulva?
\end{flashcard}

\begin{flashcard}[Vulva formation]{Determinant for tertiary cell fate}
    LIN-15 from the epidermis inhibits the formation of primary cell fates if the signal is weak. Thus P3, P4 and P8 adopts tertiary cell fate. If mutated, all cells become secondary or primary, multivulva.
\end{flashcard}

\begin{flashcard}[Partition genes]{Par-\textbf{1} mutant distribution of SKN-1, MEX-3 and GLP-1 at four cell stage...}
    All three determinants are found in all four cells.
\end{flashcard}

\begin{flashcard}[Partition genes]{Par-\textbf{2} mutant distribution of SKN-1, MEX-3 and GLP-1 at four cell stage...}
    GLP-1 is found in all four cells. 
\end{flashcard}

\begin{flashcard}[Partition genes]{Par-\textbf{3} mutant distribution of SKN-1, MEX-3 and GLP-1 at four cell stage...}
    SKN-1 and GLP-1 are found in all four cells.
\end{flashcard}
