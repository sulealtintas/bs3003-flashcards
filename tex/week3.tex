\cardfrontfoot{Week 3: \textit{C. elegans}}

\begin{flashcard}[Apoptosis]{How many cells undergo programmed cell death? How many of them are in the nervous sytem?}
    131/105
\end{flashcard}

\begin{flashcard}[Background strains used by Horvitz et al]{\textit{nuc-1} mutants}
        \textit{Nuc}lease abnormal; no DNA destruction during apoptosis. Unsuccesful
\end{flashcard}

\begin{flashcard}[Background strains used by Horvitz et al]{\textit{ced-1}/\textit{ced-2} mutants}
        No phagocytosis; apoptotic cells do not disappear. Discovered the killer gene \textit{ced3} (less cell death in mutants)
\end{flashcard}

\begin{flashcard}[Background strains used by Horvitz et al]{\textit{eg11} (gof) mutants}
        Cannot lay eggs becuase nerve cell (HSN neuron) connected to vulva dies. Discovered a new killer gene, \textit{ced4} (can lay eggs because cell death suppressed, meaning HSN neuron does not die)
\end{flashcard}

\begin{flashcard}[Background strains used by Horvitz et al]{Wild type}
    Genes discovered:
    \begin{itemize}
        \item \textit{ced-9} (gof) - NSM sister cell which usually dies survives. \textit{ced-9} functions unlike \textit{ced-3} and \textit{ced-4} in that it protects against cell death
        \item \textit{nuc-1}
        \item \textit{ced-1}
        \item \textit{egl1} (gof)
    \end{itemize}
\end{flashcard}


\begin{flashcard}[C. elegans PCD pathway]{ced-9}
    Cell death repressor. Binds \textit{ced-4}, preventing it from cleaving \textit{ced-3} which initiates PCD. Repressed by \textit{egl-1}
\end{flashcard}

\begin{flashcard}[C. elegans PCD pathway]{egl-1}
    \textbf{Egg laying deficient}. Hermaphrodite specifying neuron (HSN) dies if egl-1 has gof mutation. Cell death activator. Repressers the PCD repressor \textit{ced-9}, by forcing it to release \textit{ced-4} and starting PCD. 
\end{flashcard}

\begin{flashcard}[C. elegans PCD pathway]{ced-4}
    Cell death activator. Prepares \textit{ced-3} by cleaving it, initiating PCD. Released from \textit{ced-9} after binding \textit{egl-1}
\end{flashcard}

\begin{flashcard}[C. elegans PCD pathway]{ced-3}
    Cell death activator. Last part of the central pathway, activated by \textit{ced-4}
\end{flashcard}
