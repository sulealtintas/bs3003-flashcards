\cardfrontfoot{Week 4: \textit{Drosophila melanogaster}}

\begin{flashcard}[Fly facts]{No. of genes in Drosophila and C. elegans, no. lethal genes in Drosophila}
    13600 genes in Drosophila vs 19000 in C. elegans. 5000 lethal genes. 
\end{flashcard}

\begin{flashcard}[Oogenesis]{Oogenesis, follicle cells, oocyte}
    \textbf{Oogenesis} is the creation of the \textbf{oocyte}, the unfertilized, haploid egg cell. Follicle cells are somatic cells - the "shell" of the oocyte. 
\end{flashcard}

\begin{flashcard}[Oogenesis]{Nurse cells}
    Nurse cells are \textbf{germ line} cells. 15 nurse cells are created from one stem cell after 4 divisions, in addition to oocyte.
\end{flashcard}

\begin{flashcard}[Oogenesis]{Syncytial specification}
    Syn-cytial, same-cell. Control of individual cell specification in a cell with many nuclei, but no membranes. Occurs varying concentrations of  maternal factors (e.g. bicoid) throughout the cytoplasm. 
\end{flashcard}

\begin{flashcard}[Embryo structures]{Cephalic furrow, ventral furrow}
    Cephalic furrow is ridge in embryo that separates head from thorax. Ventral furrow eventually invaginates and creates mesoderm layer. 
\end{flashcard}

\begin{flashcard}[Embryo structures]{Segment names (head, thorax, abdomen)}
    Mx, Ma, Lb, T1-T3, A1-A8.
\end{flashcard}

\begin{flashcard}[Maternal factors]{Anterior posterior axis specification}
    First, nucleus localizes to posterior and releases Gurken mRNA close to the posterior follicle cells. Gurken protein binds to the torpedo receptor. Now bicoid and nanos can separate.
\end{flashcard}

\begin{flashcard}[Maternal factors]{Bicoid is a ... and is assisted to the anterior by ...}
    Transcription factor, assisted by exuperantia, swallow
\end{flashcard}

\begin{flashcard}[Maternal factors]{Nanos is a ... and is assisted to the posterior by ...}
    Translational repressor (hunchback). Assisted by Oskar, tudor, vasa and valois
\end{flashcard}

\begin{flashcard}[Maternal factors]{Bicoid/nanos effects on hunchback/caudal}
    Nanos-pumilio complex --| hunchback \\
    Bicoid --| caudal \\
    Bicoid --> hunchback
\end{flashcard}

\begin{flashcard}[Maternal factors]{Torso does ... and is activated by}
    Torso represses groucho, a repressor of acron/telson proteins huckebein (hkb) and tailless (tll). Activated by trunk, which is activated by torso-like protein. Torso-like only located on extremities. 
\end{flashcard}

\begin{flashcard}[Maternal factors]{Hunchback mRNA stems from...}
    Both a maternal factor and transcription/translation in the zygote. 
\end{flashcard}

\begin{flashcard}[Maternal factors]{Overexpressed bicoid results in ... \\ No bicoid results in ... \\ Inserted bicoid results in ... }
    Larger head and thorax region. \\
    No head, thorax or acron. \\
    Head, thorax and maybe acron region at insertion. 
\end{flashcard}

\begin{flashcard}[Maternal factors]{No nanos results in ... \\ No torso results in ...}
    No abdomen. \\
    No acron or telson.
\end{flashcard}
