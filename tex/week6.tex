\cardfrontfoot{Week 6: \textit{Xenopus}}

\begin{flashcard}[Experiment]{Hans Spemanns experiments}
    Hans Spemann showed that by transplanting the organizer region from a frog embryo to another, a second body w. spinal column and CNS formed. 
\end{flashcard}


\begin{flashcard}[General terms]{Blastula, blastocoel, blastomere}
    Blastula is the embryo after cleavage, but before gastrulation. Blastocoel is the empty area inside the embryo on the animal side. Blastomere is a cell in a blastula.
\end{flashcard}

\begin{flashcard}[General terms]{Gastrula(tion), Neurula(tion)}
    Gastrulation means that the germ layers have started to form (cells move inside). Neurulation means that the spinal column has started to form.
\end{flashcard}

\begin{flashcard}[General terms]{Invagination, involution, epiboly}
    Invagination is the creation of a slit (dorsal lip), involution is the movement of cells through the slit (endoderm/mesoderm layer), epiboly is the spreading of the ectoderm (skin) around the embryo as the rest of the cells move inside.
\end{flashcard}

\begin{flashcard}[General terms]{No knock-out in Xenopus due to.. but an alternative is...}
    Tetraploid genes makes it very difficult. Antisense oligos can be injected instead in case of mRNA.
\end{flashcard}

\begin{flashcard}[General terms]{Put the elements in order: \\
Grey crescent \qquad Corsical rotation \\
Gastrulation \qquad Organizer  \\
Nieuwkoop center \qquad Fertilization}
    Fertilization, corsical rotation, grey crescent, Nieuwkoop center, Organizer, gastrulation.
\end{flashcard}

\begin{flashcard}[Early development]{Mid-blastula transition (MBT) is ... and occurs at...}
    Zygotic genes are expressed. 12th cell cycle division. 
\end{flashcard}

\begin{flashcard}[Early development]{First three cell divisions, axis}
    First division divides left/right (both halves can form embryo) second division divides dorsal/ventral, third is animal/vegetal.
\end{flashcard}

\begin{flashcard}[Gastrulation]{Archenteron}
    Empty space inside the embryo created during gastrulation. Eventually becomes gut. 
\end{flashcard}

\begin{flashcard}[The dorso-ventral axis]{How is the dorsal region specified?}
Point of sperm entry specifies the ventral side. Dsh protein moves to the opposite side by cortical rotation. This becomes the Nieuwkoop center.
\end{flashcard}

\begin{flashcard}[The dorso-ventral axis]{Cortical rotation can be inhibited through...}
The chemical nocodazole, or UV-radiation, which inhibits the actin filaments. Both give a dorsalized embryo/'belly piece'.
\end{flashcard}

\begin{flashcard}[The dorso-ventral axis]{How to rescue a dorsalized embryo?}
Injection of any of the molecules found on the dorsal side: Dsh, betacatenin, simois, noggin, chordin, goosecoid... or lithium, which will inhibit GSK-3, just like Dsh.\\ Centrifugation experiments can also work.
\end{flashcard}

\begin{flashcard}[The dorso-ventral axis]{Belly pieces are the result of..}
    A ventralized embryo, which is missing the organizer.
\end{flashcard}

\begin{flashcard}[The dorso-ventral axis]{The Nieuwkoop center eventually becomes}
    Endoderm tissue. 
\end{flashcard}


\begin{flashcard}[The dorso-ventral axis]{The organizer can not form at the bottom due to...}
    The organizer is made from marginal zone ectoderm cells. 
\end{flashcard}

\begin{flashcard}[The dorso-ventral axis]{Name the proteins in the pathway resulting in the organizer, along with their function.}
    Dsh -I GSK-3 -I B-catenin -> siamois (TF) -> goosecoid 
\end{flashcard}

\begin{flashcard}[The dorso-ventral axis]{Goosecoid protein causes...}
    Movement of dorsal lip cells, induces dorsal mesodermal fate in cells, recruits nearby cells to lip 
\end{flashcard}

\begin{flashcard}[Induction of the mesoderm]{Mesoderm cells are the result of ...}
Endoderm cells inducing ectoderm cells at the marginal zone by releasing Vg-1. 
\end{flashcard}

\begin{flashcard}[Induction of the mesoderm]{Maternal factors in the vegetal pole}
VegT, Vg1.
\end{flashcard}

\begin{flashcard}[Induction of the mesoderm]{VegT activates...}
Derriere, Xnrf, Vg-1
\end{flashcard}

\begin{flashcard}[Induction of the mesoderm]{Function of Vg-1, Derriere, Xnrf}
    Transforms ectoderm cells to mesoderm cells when received, by activating Xbra in mesoderm cells, determining their fate.
\end{flashcard}


%%%%%%%%%%%%%%%%%%%%%%%%%%%%%%%%%%%%%%%%%%%%%%%%%%%%%%%%%%%%%%%%%%%%%%%%%%%%%%
% noget om Xbra og hvordan det er endoderm der inducerer (ventral) mesoderm...
%%%%%%%%%%%%%%%%%%%%%%%%%%%%%%%%%%%%%%%%%%%%%%%%%%%%%%%%%%%%%%%%%%%%%%%%%%%%%%


\begin{flashcard}[Organiser]{What's required for siamois to activate goosecoid and form the organiser?}
High levels of Xnrf released from the Nieuwkoop center, meaning only ectodermal cells can form dorsal mesoderm/organiser.
\end{flashcard}

\begin{flashcard}[Organiser]{Which gene indicates the presence of the organiser?}
Goosecoid.
\end{flashcard}

\begin{flashcard}[Organiser]{For dorsal mesoderm specification, you need...}
B-catenin. Usually depleted by GSK-3, but Dsh protein, which is found in the dorsal region, inhibits GSK-3, allowing Betacatenin to bind to Tcf-3 and transform it from a repressor to an activator of simois.
\end{flashcard}

\begin{flashcard}[BMP-4 in mesodermal differentiaion]{High levels of BMP specify...}
Blood.
\end{flashcard}

\begin{flashcard}[BMP-4 in mesodermal differentiaion]{Intermediate levels of BMP specify...}
Kidneys, muscle, heart.
\end{flashcard}

\begin{flashcard}[BMP4 in mesodermal differentiaion]{Low levels of BMP4 specify...}
Notochord.
\end{flashcard}

\begin{flashcard}[BMP4 in mesodermal differentiaion]{BMP4 is inhibited by...}
Noggin, chordin (and frizzzzzzbee) from the organiser.
\end{flashcard}
