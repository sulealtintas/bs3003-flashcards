\cardfrontfoot{Week 7: \textit{Xenopus}}

%%% Maybe smth about mouse and chick node/primary streak etc 

\begin{flashcard}[Neurulation]{Paraxial mesoderm essentially means}
    Cells that form somites, which forms trunk and limb muscles, and ribcage. 
\end{flashcard}

\begin{flashcard}[Neurulation]{The notochord is responsible for...}
    Forming the neural plate, eventually becomes spine, brain and CNS.
\end{flashcard}

\begin{flashcard}[Neurulation]{The neural plate becomes the neural tube by..}
    Neural folds folding over, "zipping" the neural plate up. 
\end{flashcard}


\begin{flashcard}[Neurulation]{The part of the ectoderm that forms neurons is ..}
    The part closest to the organizer after gastrulation.
\end{flashcard}

\begin{flashcard}[Neurulation]{Ectodermic cells isolated form ... due to ..}
    Neurons, as the community effect with BMP is necessary for epidermis to form.
\end{flashcard}

\begin{flashcard}[General]{Embryos look similar at the ... stage}
    Phylotypic stage, late part of organogenesis.
\end{flashcard}

\begin{flashcard}[AP axis in Neurulation]{Neural tissue is induced by ... and differentiated by ...}
    planar signals (along the surface), lateral signals (from the notochord).
\end{flashcard}

\begin{flashcard}[AP axis in Neurulation]{Posterior ectoderm is induced by high amounts of }
    Wnt, FGF, retinoic acid (RA).
\end{flashcard}

\begin{flashcard}[AP axis in Neurulation]{Trunk is induced by high amounts of }
    Inhibitors of BMP chordin, noggin and follistatin
\end{flashcard}

\begin{flashcard}[AP axis in Neurulation]{Head is induced by high amounts of }
    Wnt inhibitors Cerberus, frizzbee, dickkopf plus IGF (insulin growth factor).
\end{flashcard}

\begin{flashcard}[AP axis in Neurulation]{High amounts of retinoic acid causes...}
    Posteriorization of embryo.
\end{flashcard}
